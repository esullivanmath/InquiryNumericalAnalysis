\chapter{Building PDE's From Conservation Laws}

In this section
we'll give a more analytic introduction to most of the primary partial
differential equations of interest in basic mathematical physics.  We will make reference
to Fick's Law for mass transport and Fourier's Law for thermal transport, so interested
readers should dig deeper by examining the relevant Wikipedia pages or other sources.
  
Conservation laws pervade all of physics -- conservation of energy, conservation of
momentum, and conservation of mass.  These laws are sometimes stated colloquially as
{\it energy (or momentum or mass) can neither be created nor destroyed}, but this phrase
is not super helpful mathematically.  We start this section with a brief mathematical
derivation of a {\it general conservation law} to further clarify what we mean
mathematically.  The resulting general conservation law will be a
partial differential equation that can be used to mathematically express the physical laws
of conservation of mass, momentum, or
energy.

Let $u$ be the quantity you are trying to conserve, $\bq$ be the flux of that quantity,
and $f$ be any source of that quantity.  For example, if we are to derive a conservation
of energy equation, $u$ might be energy, $\bq$ might be temperature flux, and $f$ might be
a temperature source (or sink).

\subsection*{Derivation of General Balance Law}
Let $\Omega$ be a fixed volume and denote the boundary of this volume by $\partial
\Omega$. The rate at which $u$ is changing in time throughout $\Omega$ needs to be
balanced by the rate at which $u$ leaves the volume plus any sources of $u$.
Mathematically, this means that
\begin{flalign}
    \pd{ }{t} \iiint_{\Omega} u dV = -\iint_{\partial \Omega} \bq \cdot n dA +
    \iiint_\Omega f dV.
    \label{eqn:global_balance}
\end{flalign}
This is a global balance law in the sense that it holds for all volumes $\Omega$.  The
mathematical 
troubles here are two fold: (1) there are many integrals, and (2) there are really two variables
($u$ and $q$ since $f=f(u,x,t)$) so the equation is not closed.  In order to mitigate
that fact we apply the divergence theorem to the first term on the right-hand side of
\eqref{eqn:global_balance} to get
\begin{flalign}
    \pd{ }{t} \iiint_{\Omega} u dV = -\iiint_{\Omega} \nabla \cdot \bq dV +
    \iiint_\Omega f dV.
    \label{eqn:global_balance2}
\end{flalign}

Gathering all of the terms on the right of \eqref{eqn:global_balance2}, interchanging the integral and the derivative on
the left (since the volume is not changing in time), and rewriting gives
\begin{flalign}
    \iiint_\Omega \left( \pd{u}{t} + \nabla \cdot \bq \right) dV = \iiint_\Omega f dV
    \label{eqn:global_balance3}
\end{flalign}
If we presume that this equation holds for all volumes $\Omega$ then the integrands must
be equal and we get the local balance law
\begin{flalign}
    \pd{u}{t} + \nabla \cdot \bq = f.
    \label{eqn:local_balance}
\end{flalign}

Equation \eqref{eqn:local_balance} is an expression of the balances of changes in time to
changes in space of a conserved quantity such as mass, momentum, or energy.  What remains
is to make clear the meaning and functional form of the flux $\bq$ and the source function
$f$.

\subsection*{Simplification of the Local Balance Law}
In equation \eqref{eqn:local_balance} it is often assumed that the system is free of
external sources.  In this case we set $f$ to zero and obtain the source-free balance law
\begin{flalign}
    \pd{u}{t} + \nabla \cdot \bq = 0.
    \label{eqn:local_source_free}
\end{flalign}
It is this form of balance law where many of the most interesting and important partial
differential equations come from.  In particular consider the following two cases: mass
balance and energy balance.
\subsection*{Mass Balance}
In mass balance we take $u$ to either be the density of a substance (e.g. in the case of
liquids) or the concentration of a substance in a mixture (e.g. in the case of
gasses). If $C$ is the mass concentration of a substance in a gas then the flux of that
substance is given via Fick's Law as
\begin{flalign}
    \bq = -k \nabla C.
    \label{eqn:fick}
\end{flalign}
Combining \eqref{eqn:fick} with \eqref{eqn:local_source_free} (and assuming that $k$ is
independent of space, time, and concentration) gives
\begin{flalign}
    \pd{C}{t} = k \nabla \cdot \nabla C. 
    \label{eqn:fick2_simp}
\end{flalign}
In the presence of external sources of mass, \eqref{eqn:fick2_simp} is
\begin{flalign}
    \pd{C}{t} = k \nabla \cdot \nabla C + f(x).
    \label{eqn:fick3}
\end{flalign}
Expanding the Laplacian operator on the right-hand side of \eqref{eqn:fick3} we get
\begin{flalign}
    \pd{C}{t} = k\left( \pdd{C}{x} + \pdd{C}{y} + \pdd{C}{z} \right) + f(x)
    \label{eqn:fick3_expanded}
\end{flalign}
where the reader should note that this can be easily simplified in 1 or 2 spatial
dimensions.
% \begin{problem}
%     What does \eqref{eqn:fick3} equation look like in terms of spatial derivatives on the
%     right-hand side?
%     \begin{flalign*}
%         \pd{C}{t} &= \underline{\hspace{2in}} \quad \text{(1 Spatial Dimension)} \\
%         \pd{C}{t} &= \underline{\hspace{2in}} \quad \text{(2 Spatial Dimensions)} \\
%         \pd{C}{t} &= \underline{\hspace{2in}} \quad \text{(3 Spatial Dimensions)}
%     \end{flalign*}
% \end{problem}

\subsection*{Energy Balance}
The energy balance equation is essentially the same as the mass balance equation.  If $u$
is temperature then the flux of temperature is given by Fourier's Law for heat conduction
\begin{flalign}
    \bq = -k\nabla T.
    \label{eqn:fourier}
\end{flalign}
Making the same simplifications as in the mass balance equation we arrive at
\begin{flalign}
    \pd{T}{t} = k \nabla \cdot \nabla T.
    \label{eqn:fourier2}
\end{flalign}
In the presence of external sources of heat, \eqref{eqn:fourier2} becomes
\begin{flalign}
    \pd{T}{t} = k \nabla \cdot \nabla T + f(x).
    \label{eqn:fourier3}
\end{flalign}
Expanding the Laplacian operator on the right-hand side of \eqref{eqn:fourier3} we get
\begin{flalign}
    \pd{T}{t} = k\left( \pdd{T}{x} + \pdd{T}{y} + \pdd{T}{z} \right) + f(x)
    \label{eqn:fourier3_expanded}
\end{flalign}
where the reader should note that this can be easily simplified in 1 or 2 spatial
dimensions.
% \begin{problem}
%     What does \eqref{eqn:fourier3} equation look like in terms of spatial derivatives on the
%     right-hand side?
%     \begin{flalign*}
%         \pd{T}{t} &= \underline{\hspace{2in}} \quad \text{(1 Spatial Dimension)} \\
%         \pd{T}{t} &= \underline{\hspace{2in}} \quad \text{(2 Spatial Dimensions)}\\
%         \pd{T}{t} &= \underline{\hspace{2in}} \quad \text{(3 Spatial Dimensions)}
%     \end{flalign*}
% \end{problem}



\subsection*{Laplace's Equation and Poisson's Equation}
Equations \eqref{eqn:fick3} and \eqref{eqn:fourier3} are the same partial differential
equation for two very important physical phenomenon; mass and heat transfer.  In the case
where time is allowed to run to infinity and no external sources of mass or energy are
included these equations reach a steady state solution (no longer changing in time) and we
arrive at Laplace's Equation
\begin{flalign}
    \nabla \cdot \nabla u = 0.
    \label{eqn:laplace}
\end{flalign}
Laplace's equation is actually a statement of minimal energy as well as steady state heat
or temperature.  We can see this since entropy always drives systems from high energy to
low energy, and if we have reached a steady state then we must have also reached a surface
of minimal energy.

Equation \eqref{eqn:laplace} is sometimes denoted as $\nabla \cdot \nabla u = \nabla^2 u =
\Delta u$, and in terms of the partial derivatives it is written as
\begin{flalign*}
    \pdd{u}{x} + \pdd{u}{y} + \pdd{u}{z} = 0.
% V    0 &= \underline{\hspace{2in}} \quad \text{(1 Spatial Dimension)} \\
%     0 &= \underline{\hspace{2in}} \quad \text{(2 Spatial Dimensions)} \\
%     0 &= \underline{\hspace{2in}} \quad \text{(3 Spatial Dimensions)} 
\end{flalign*}

If there is a time-independent external source the right-hand side of
\eqref{eqn:laplace} will be non-zero and we arrive at Poisson's equation:
\begin{flalign}
    \nabla \cdot \nabla u = -f(x).
    \label{eqn:poisson}
\end{flalign}
Note that the negative on the right-hand side comes from the fact that
$\pd{u}{t} = k \nabla \cdot \nabla u + f(x)$ and $\pd{u}{t} \to 0$.  Technically we are
absorbing the constant $k$ into $f$ (that is ``$f$'' is really ``$f/k$'').  Also
note that in many instances the value of $k$ is not constant and cannot therefore be pulled
out of the derivative without a use of the product rule.

Let's summarize:
\begin{center}
    \begin{tabular}{|c|c|c|}
        \hline
        Name of PDE & PDE & What the PDE Models \\ \hline \hline
        The Heat Equation & $\ds \pd{u}{t} = k \nabla \cdot \nabla u + f(x)$ & Diffusion \\
        Laplace's Equation & $\ds k \nabla \cdot \nabla u =-f(x)$ & Minimal Energy
        Surfaces \\
%         The Wave Equation & $\ds \pdd{u}{t} = k \nabla \cdot \nabla u + f(x)$ & Wave
%         phenomena \\
        \hline
    \end{tabular}
\end{center}

Further discussion of the origins of the wave equation and other interesting PDE's is left
to the reader.

