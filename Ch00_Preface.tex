\setcounter{chapter}{-1}
\chapter{To the Student and the Instructor}
This document contains lecture notes, classroom activities, code, examples, and challenge
problems specifically designed for an introductory semester of numerical analysis.  The content
herein is written and maintained by Dr. Eric Sullivan of Carroll College.  Problems were
either created by Dr. Sullivan, the Carroll Mathematics Department faculty, part of NSF
Project Mathquest, or come from other sources and are either cited directly or cited in
the \LaTeX\,source code for the document (and are hence purposefully invisible to the
student).


\section{An Inquiry Based Approach}
This material is written with an Inquiry-Based Learning (IBL) flavor. In that sense, this
document could be used as a stand-alone set of materials for the course but these notes
are not a {\it traditional textbook} containing all of the expected theorems, proofs,
code, examples, and exposition. The students are encouraged to work through problems and
homework, present their findings, and work together when appropriate. You will find that
this document contains collections of problems with only minimal interweaving exposition.
It is expected that you do every one of the problems and then use other more traditional
texts as a backup when you are stuck.  Let me say that again: this is not the only set of
material for the course.  Your brain, your peers, and the books linked in the next section
are your best resources when you are stuck.

To learn more about IBL go to
\href{http://www.inquirybasedlearning.org/about/}{http://www.inquirybasedlearning.org/about/}.
The long and short of it is that the students in the class are the ones that are doing the
work; proving theorems, writing code, working problems, leading discussions, and pushing
the pace. The instructor acts as a guide who only steps in to redirect conversations or to
provide necessary insight. If you are a student using this material you have the following
jobs:
\begin{enumerate}
\item Fight!  You will have to fight hard to work through this material.  The fight is
        exactly what we're after since it is ultimately what leads to innovative thinking.
\item Screw Up!  More accurately, don't be afraid to screw up.  You should write code,
    work problems, and prove theorems then be completely unafraid to scrap what you've
    done and redo it from scratch.  Learning this material is most definitely a non-linear
    path.
\item Collaborate!  You should collaborate with your peers with the following caveats:
    \begin{enumerate}
        \item When you are done collaborating you should go your separate ways.  When you
            write your solution you should have no written (or digital) record of your
            collaboration.  
        \item \underline{The internet is not a collaborator}.  Use of the internet to help
            solve these problems robs you of the most important part of this class; the
            chance for original though.
    \end{enumerate}
\item Enjoy!  Part of the fun of IBL is that you get to experience what it is like to
        think like a true mathematician / scientist.  It takes hard work but ultimately
        this should be fun!
\end{enumerate}

\section{Online Texts and Other Resources}\label{pref:resources}
If you are looking for online textbooks for linear algebra and differential equations I
can point you to a few.  Some of the following online resources may be a good place to
help you when you're stuck but
they will definitely say things a bit differently. Use these resources wisely.
\begin{itemize}
    \item Holistic Numerical Methods
        \href{http://nm.mathforcollege.com/}{http://nm.mathforcollege.com/}\\
        The Holistic Numerical Methods book is probably the most complete free reference
        that I've found on the web.  This should be your source to look up deeper
        explanations of problems, algorithms, and code.
    \item Scientific Computing with MATLAB
        \href{http://gribblelab.org/scicomp/scicomp.pdf}{http://gribblelab.org/scicomp/scicomp.pdf}
    \item Tea Time Numerical Analysis
        \href{http://lqbrin.github.io/tea-time-numerical/}{http://lqbrin.github.io/tea-time-numerical/}
\end{itemize}


\section{To the Instructor}
If you are an instructor wishing to use these materials then I only ask that you adhere to the
Creative Commons license.  You are welcome to use, distribute, and remix these materials
for your own purposes.  Thanks for considering my materials for your course!

