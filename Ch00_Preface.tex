% \setcounter{chapter}{-1}
\frontmatter
\chapter{Preface}
This book grew out of lecture notes, classroom activities, code, examples, exercises,
projects, and challenge problems for my introductory course on
numerical methods.  The prerequisites for this material include a firm understanding of
single variable calculus (though multivariable calculus doesn't hurt), a good
understanding of the basics of linear algebra, a good understanding of the basics of
differential equations, and some exposure to scientific computing (as seen in other math
classes or perhaps from a computer science class). The primary audience is any
undergraduate STEM major with an interest in using computing to solve problems.

A note on the book's title: I do not call these materials ``numerical
analysis'' even though that is often what this course is called.  In these materials I
emphasize ``methods'' and implementation over rigorous mathematical ``analysis.''  While
this may just be semantics I feel that it is important to point out.  If you are looking
for a book that contains all of the derivations and rigorous proofs of the primary results
in elementary numerical analysis, then this not the book for you.  I have
intentionally written this material with an inquiry-based emphasis which means that this
is not a traditional text on numerical analysis -- there are plenty of those on the market
and I cite several wonderful traditional texts in the bibliography at the end of this
book.

\section{To The Student}

\subsection{The Inquiry-Based Approach}
% % The inquiry-based approach can by summarized by one quote:
% \begin{quote}
%     {\it Any creative endeavor is built on the ash heap of failures.}
% \end{quote}
% The material in this book is meant to make you think, build, construct, fail, struggle,
% and ultimately succeed in learning numerical methods.

\begin{problem}[Setting The Stage]
    Let's start the book off right away with a problem designed for groups, discussion,
    disagreement, and deep critical thinking.  This problem is inspired by Dana Ernst's
    first day IBL activity titled: \href{http://danaernst.com/setting-the-stage/}{Setting
    the Stage}.
    \begin{itemize}
        \item Get in groups of size 3-4.
        \item Group members should introduce themselves.
        \item For each of the questions that follow I will ask you to:
            \begin{enumerate}
                \item {\bf Think} about a possible answer on your own
                \item {\bf Discuss} your answers with the rest of the group
                \item {\bf Share} a summary of each group's discussion
            \end{enumerate}
    \end{itemize}
    {\bf Questions:} 
    \begin{description}
        \item[Question \#1:] What are the goals of a university education?
        \item[Question \#2:] How does a person learn something new?
        \item[Question \#3:] What do you reasonably expect to remember from your courses
            in 20 years?
        \item[Question \#4:] What is the value of making mistakes in the learning process?
        \item[Question \#5:] How do we create a safe environment where risk taking is
            encouraged and productive failure is valued?
    \end{description}
\end{problem}


This material is written with an Inquiry-Based Learning (IBL) flavor. In that sense, this
document could be used as a stand-alone set of materials for the course but these notes
are not a {\it traditional textbook} containing all of the expected theorems, proofs,
code, examples, and exposition. You are encouraged to work through problems and homework,
present your findings, and work together when appropriate. You will find that this
document contains collections of problems with only minimal interweaving exposition.  It
is expected that you do every one of the problems and then only use other more traditional
texts (or Google) as a backup when you are completely stuck.  Let me say that again: this
is not the only set of material for the course.  Your brain, your peers, and the books
linked in the next section are your best resources when you are stuck.

To learn more about IBL go to
\href{http://www.inquirybasedlearning.org/about/}{http://www.inquirybasedlearning.org/about/}.
The long and short of it is that you, the student, are the one that is doing the
work; proving theorems, writing code, working problems, leading discussions, and pushing
the pace. The instructor acts as a guide who only steps in to redirect conversations or to
provide necessary insight. 

You have the following jobs as a student in this class:
\begin{enumerate}
    \item {\bf Fight!}  You will have to fight hard to work through this material.  The fight is
        exactly what we're after since it is ultimately what leads to innovative thinking.
    \item {\bf Screw Up!}  More accurately, don't be afraid to screw up.  You should write code,
    work problems, and prove theorems then be completely unafraid to scrap what you've
    done and redo it from scratch.  Learning this material is most definitely a non-linear
    path.
\item {\bf Collaborate!}  You should collaborate with your peers with the following caveats:
    \begin{enumerate}
        \item When you are done collaborating you should go your separate ways.  When you
            write your solution you should have no written (or digital) record of your
            collaboration.  
        \item \underline{The internet is not a collaborator}.  Use of the internet to help
            solve these problems robs you of the most important part of this class; the
            chance for original thought.
    \end{enumerate}
\item {\bf Enjoy!}  Part of the fun of IBL is that you get to experience what it is like to
        think like a true mathematician / scientist.  It takes hard work but ultimately
        this should be fun!
\end{enumerate}

\subsection{Online Texts and Other Resources}\label{pref:resources}
If you are looking for online textbooks for numerical methods or numerical analysis I can
point you to a few of my favorites.  Some of the following online resources may be a good
place to help you when you're stuck but they will definitely say things a bit differently.
Use these resources wisely.
\begin{itemize}
    \item Holistic Numerical Methods
        \href{http://nm.mathforcollege.com/}{http://nm.mathforcollege.com/}\\
        The Holistic Numerical Methods book is probably the most complete free reference
        that I've found on the web.  This should be your source to look up deeper
        explanations of problems, algorithms, and code.
    \item Scientific Computing with MATLAB
        \href{http://gribblelab.org/scicomp/scicomp.pdf}{http://gribblelab.org/scicomp/scicomp.pdf}
    \item Tea Time Numerical Analysis
        \href{http://lqbrin.github.io/tea-time-numerical/}{http://lqbrin.github.io/tea-time-numerical/}
\end{itemize}


\section{To the Instructor}
If you are an instructor wishing to use these materials then I only ask that you adhere to the
Creative Commons license.  You are welcome to use, distribute, and remix these materials
for your own purposes.  Thanks for considering my materials for your course!  Let me know
if you have questions, edits, or suggestions: esullivan@carroll.edu.

\subsection{The Inquiry-Based Approach}
I have written these materials with an inquiry-based flavor.  This means that this is not
a traditional textbook.  I don't
lecture through hardly any of the material in the book.  Instead my classes are structured
so that students are given problems to work before class, we build off of those problems
in class, and we repeat.  The exercises at the end of the chapters are assigned weekly and
graded with a revision process in mind -- students redo problems if the coding was
incorrect, if the mathematics was incorrect, or if they somehow missed the point.  The
students are tasked with building most of the algorithms, code, intuition, and analysis
with my intervention only if I deem it necessary. 

Several of the problems throughout the book are meant to be done in groups either at
the boards in the classroom or in some way where they can share their work.  Much of my
class time is spent with students actively building algorithms or group coding.  The
beauty, as I see it, of IBL is that you can run your course in any way that is comfortable
for you.  You can lecture through some of the material in a more traditional way, you can
let the students completely discover some of the methods, or you can do a mix of both.

You will find that I do not give rigorous (in the mathematical sense) proofs or
derivations of many of the algorithms in this book.  I tend to lean on
numerical experiments to allow students to discover algorithms, error estimates, and other
results without the rigor.
The makeup of my classes tends to be math majors along with engineering, computer science,
physics, and data science students.  While the math majors need to eventually see the
rigorous derivations the other majors, I think, are better served working from an
experimental approach rather than a theorem-proof approach.

\subsection{The Projects}
I have taught this class with anywhere from two to four projects during the semester.
Each of the projects is designed to give the students an open-ended task where they can
show off their coding skills and, more importantly, build their mathematical communication
skills.  Projects can be done in groups or individually depending on the background and
group dynamics of your class.  Appendix \ref{app:writing_projects} contains several tips
for how to tackle the writing in the projects.  Appendix \ref{app:latex} gives students
tips for writing in \LaTeX\ if indeed you want your students using that tool.  

\subsection{Coding}
I expect that my students come with some coding experience from other mathematics or
computer science classes.  With that, I leave the coding help as an appendix (see Appendix
\ref{app:coding}) and only point the students there for refreshers.  If your students need
a more thorough ramp up to the coding then you might want to start with Appendix
\ref{app:coding} to get the students up to speed.  I expect the students to do most of the
coding the in the class, but occasionally we will code algorithms together (especially
earlier in the semester when the students are still getting their feet underneath them).

\ifnum\Python=0 If students are coding in MATLAB then be sure that all students have
access.  We have a site license that students can log into via a virtual desktop.  A
student license tends to serve students well. \fi
\ifnum\Python=1 I encourage students to learn Python.  It is a general purpose language
that does extremely well with numerical computing when paired with \texttt{numpy} and
\texttt{matplotlib}.  Appendix \ref{app:coding} has several helpful sections for getting
students up to speed with Python.

I encourage you to consider having your students code in Jupyter Notebooks.  The advantage
is that students can mix their writing and their code in a seamless way.  This allows for
an iterative approach to coding and writing and gives the students the tools to explain
what they're doing as they code.
\fi

\subsection{Pacing}
The following is a typical 15-week semester with these materials.
\begin{itemize}
    \item Chapter 1 - 1.5 weeks
    \item Chapter 2 - 1.5 weeks
    \item Chapter 3 - 2 weeks
    \item Chapter 4 - 3 weeks
    \item Chapter 5 - 3 weeks
    \item Chapter 6 - 3 weeks
\end{itemize}

\subsubsection*{Other Considerations:}
\begin{description}
    \item[Projects:] I typically assign a project after Chapter 2 or 3, a second project
        after Chapter 4, and a third project after Chapter 5.  The fourth project, if time
        allows, typically comes from Chapter 6.  I typically dedicate two class days to
        the first project and then one class day to each subsequent project. For the final
        project I typically have students present their work so this takes a day or two
        out of our class time. 
    \item[Exercises:] I typically assign one collection of exercises per week.  Students
        are to work on these outside of class, but in some
        cases it is worth taking class time to let students work in teams.  Of particular
        note are the coding exercises in Chapter \ref{ch:intro}.  If your students need
        practice with coding then it might be worthwhile to mix these exercises in through
        several assignments and perhaps during a few class periods.  I have also taken
        extra class time with the exercises in Chapter \ref{ch:odes} to let the students
        work in pairs on the modeling aspects of some of the problems.
    \item[Exams:] This is a non-traditional book and as such you might want to consider
        some non-traditional exam settings.  Some ideas that my colleagues and I have used
        are: 
        \begin{itemize}
            \item Use code and functions that you've written to solve several new
                problems during a class period.
            \item Give the mathematical details and the derivations of key algorithms.
            \item Take several problems home (under strict rules about collaboration) and
                return with working code and a formal write up.
            \item No exams, but put heavier weight on the projects.
        \end{itemize}
\end{description}

\section{Special Thanks}
I would first like to thank Dr. Kelly Cline for being brave enough to teach a course that
he loves out of a rough draft of my notes.  Your time, suggested edits, and thoughts for
future directions of the book were, and are, greatly appreciated.  Second, I would like to
thank Johnanna for simply being awesome.  Next I would like to thank the institution
of Carroll College for seeing this project as a worthy academic pursuit even though the
end result is not a book or publication in the traditional sense.  Finally, I would like
to thank all of my colleagues and students, both past and present, in the math department
at Carroll.  The suggestions, questions, struggles, and triumphs of these folks are what
have shaped this work into something that I'm proud of and that I hope will be a useful
resource for future students and instuctors.
