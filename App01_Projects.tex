\chapter{Writing and Projects}
This class is writing intensive and as such you will be writing several papers.  This
appendix is designed to give you helpful hints for the writing.

\section{The Paper} 
Write your work in a formal paper that is typed and written at a college level using
appropriate mathematical typesetting. This paper must be organized into sections, starting
with a Summary or Abstract, followed by an Introduction, and ending with Conclusions and
References. Each of these sections should begin with these headings in a large bold font
(using the \LaTeX\ \verb|\section| and \verb|\subsection| commands where appropriate).
Within sections I would suggest using subheadings to further organize things and aid in
clarity.

\section{Figures and Tables}
Figures and tables are a very important part of these projects. Never break tables or
figures across pages. Each figure or table must fit completely onto one sheet of paper. If
your table has too much information to fit onto one sheet, divide it into two separate
tables. In addition to the figure, this sheet must contain the figure number, the figure
title, and a brief caption; for example ``Figure 2: A plot of heating oil price versus time
from Model F1. We see that the effects of seasonal variation in price are dominated by
random fluctuations.'' In the text, refer to the figure/table by its number. For example in
the text you might say ``As we see in Figure 2, in model F1 the effects of seasonal 
variation in price are dominated by random fluctuations.'' Every figure and table must be
mentioned (by number) somewhere in the text of your paper. If you do not refer to it
anywhere in the text, then you do not need it, and subsequently it will be ignored.

Think of figures and tables as containing the evidence that you are using to support the
point you are trying to make with your paper. Always remember that the purpose of a figure
or a table is to show a pattern, and when someone looks at the figure this pattern should
be obvious. Figures should not be cluttered and confusing: They should make things very
clear. Always label the horizontal and vertical axes of plots.

\section{Writing Style}
The real goal of mathematical writing is to take a complex and intricate subject and to
explain it so simply and so plainly that the results are obvious for everyone. I want your
paper to demonstrate that you not only did the right calculations, but that you understand
what you did and why your methods worked.

Write this paper using the word `we' instead of `I.' For example: ``First we calculate the
sample mean.'' This `we' refers to you and the reader as you guide the reader through the
work that you've done. Also please avoid the word ``prove'' or ``proof.'' Numerical methods
usually deal with approximations, not absolutes, and in mathematics we reserve the word
``prove'' for things that are absolutely 100\% certain. Often the word ``test'' can be used
instead of ``prove.''



\section{Tips For Writing Clear Mathematics}
At this point you know just enough mathematics and \LaTeX\ to be dangerous.  It is time to
clean up your act and teach you some of the formalities about writing mathematics.  The
following sections stem from a document that I give all of my upper level mathematics
students.  

Some tips for writing clear mathematics can be found here: \\
\href{http://www.ohio.edu/people/mohlenka/goodproblems/goodstudent.html}{http://www.ohio.edu/people/mohlenka/goodproblems/goodstudent.html}


\subsection{Audience}
The following suggestions will help you to submit properly written homework solutions,
papers, projects, labs, and proofs. The goal of any writing is to clearly communicate
ideas to another person. Remember that the other person may even be your future self. When
you write for another person, you will need to include ideas that may be in your mind but
omitted when you are writing a rough draft on scratch paper. If you keep your intended
audience in mind, you will produce higher quality work. For a course in mathematics, the
intended audience is usually your instructor, your classmates, or a student grader. This
implies that your task is to show that you thoroughly understand your solution.
Consequently, you should routinely include more details.

One rule of thumb must prevail throughout all mathematical writing:\index{writing!rules}
\begin{quote}
    When you read a mathematical solution out loud it needs to make sense as
    grammatically correct English writing.  This includes reading all of the symbols with
    the proper language.
    \begin{center}
        {\bf Don't forget that mathematics is a language that is meant to be spoken and
        read just like works of literature!}
    \end{center}
\end{quote}


\subsection{How To Make Mathematics Readable -- 10 Things To Do}
\begin{enumerate}
    \item When read aloud, the text and formulas must form complete English sentences. If
        you get lost, say aloud what you mean and write it down.
    \item Every mathematical statement must be complete and meaningful. Avoid fragments.
    \item If a statement is something you want to prove or something you assume
        temporarily, e.g., to discuss possible cases or to get a contradiction, say so
        clearly. Otherwise, anything you put down must be a true statement that follows
        from your up front assumptions.
    \item Write what your plan is.  It will also help you focus on what to do.
    \item There must be sufficient detail to verify your argument. If you do not have the
        details, you have no way of knowing if what you wrote is correct or not. Keep the
        level of detail uniform.
    \item If you are not sure, even slightly, about something, work out the details on the
        side with utmost honesty, going as deep as necessary. Decide later how much detail
        to include. 
    \item Do not write irrelevant things just to fill paper and show you know something.
    \item Your argument should flow well. Make the reading easy. Logical and intuitive
        notation matters.
    \item Keep in mind what the problem is and make sure you are not doing something else.
        Many problems are solved and proofs done simply by understanding what is what.
    \item The state of mind when you are inventing a solution is completely different from
        the mode of work when you are writing the solution down and verifying it.  Learn
        how to go back and forth between the two. The act of typing your solutions forces
        you to iterate over this process but remember that the process isn't done until
        you've proofread what you typed.
\end{enumerate}



\subsection{Some Writing Tips}\index{writing!tips}
\begin{description}
    \item[Use sentences:] The feature that best distinguishes between a properly
        written mathematical exposition and a piece of scratch paper is the use (or
        lack) of sentences. Properly written mathematics can be read in the same manner as
        properly written sentences in any other discipline. Sentences force a linear
        presentation of ideas. They provide the connections between the various
        mathematical expressions you use. This linearity will also keep you from handing
        in a page with randomly scattered computations with no connections. The
        sentences may contain both words and mathematical expressions. Keep in mind that
        the way your present your solution may be different than the way that you arrived
        at the solution.  It is imperative that you work problems on scratch paper first
        before formally writing the solution.  
        
        The following extract illustrates these ideas.

        \begin{quote}
           Let $n$ be odd. Then Definition 3.10 indicates that there does not exist an
           integer, $k$, such that $n = 2k$. That is, $n$ is not divisible by $2$. The
           Quotient– Remainder theorem asserts that $n$ can be uniquely expressed in the
           form $n = 2q + r$ , where $r$ is an integer with $0 \le r < 2$. Thus, $r \in
           \{0, 1\}$. Since $n$ is not divisible by 2, the only admissible choice is $r =
           1$. Thus, $n = 2q + 1$, with $q$ an integer.
        \end{quote}


    \item[Read out loud:] The sentences you write should read well out loud. This will
        help you to avoid some common mistakes. Avoid sentences like:
        \begin{quote}
            Suppose the graph has $n$ number of vertices.\\
            The piggy bank contains $n$ amount of coins.
        \end{quote}
        If you substitute an actual number for $n$ (such as 4 or 6) and read these out loud
        they will sound wrong (because they are wrong). The variable $n$ is already a
        numeric variable so it should be read just like an actual number. The correct
        versions are:
        \begin{quote}
            Suppose the graph has $n$ vertices.\\
            (Read this as: ``Suppose the graph has en vertices''.) \\
            The piggy bank contains $n$ coins.
        \end{quote}

        You should also avoid sentences like:
        \begin{quote}
            From the previous computation $x=5$ is true.
        \end{quote}
        A better way to say this is:
        \begin{quote}
            From the previous computation we see that $x=5$.
        \end{quote}
        When you read the equal sign as part of the sentence you realize that there
        is no reason to write ``is true''.


    \item[$=$ is NOT a conjunction:] The mathematical symbol $=$ is an assertion that
        the expression on its left and the expression on its right are equal. Do not use
        it as a connection between steps in a series of calculations. Use words for this
        purpose.  Here is an example that misuses the $=$ symbol when solving the equation
        $3x=6$:
        \[ \text{ {\bf Incorrect!} } \qquad 3x = \underbrace{6 =
            \frac{3x}{3}}_{\text{false!}} = \frac{6}{3} = x = 2 \]
        One proper way to write his is: \\
        $3x=6$.  Dividing both sides by $3$ leads to $\frac{3x}{3} =
        \frac{6}{3}$, which simplifies to $x=2$.

    \item[``$\implies$'' means ``implies'':]  The double arrow ``$\implies$'' means
        that the statement on the left logically implies the statement on the right.  This
        symbol is often misused in place of the ``$=$'' sign. 

    \item[Do not merge steps:] Suppose you need to calculate the final price for a
        \$20 item with 7\% sales tax. One strategy is to first calculate the tax, then add
        the \$20. Here is an incorrect way to write this. \\
        \[ \text{ {\bf Incorrect!} } \qquad  20 \cdot 0.07 \underbrace{=}_{false!} 1.4 +
            20 \underbrace{=}_{false!} \$21.4. \]
        The main problem (besides the magically-appearing dollar sign at the end) is that
        $20 \cdot 0.07 \neq 1.4 + 20$. The writer has taken the result of the
        multiplication ($1.4$) and merged directly into the addition step, creating a lie
        (since $1.4\neq 21.4$). The calculations could be written as:
        \[ \$20 \cdot 0.07 = \$1.40 \text{so the total price is } \$1.40 + \$20 = \$21.40
            \]

        \item[Avoid ambiguity:] When in doubt, repeat a noun rather using unspecific words
        like ``it'' or ``the''. For example, in the sentences
        \begin{quote}
            Let $G$ be a simple graph with $n \ge 2$ vertices that is not complete and let
            $G$ be its complement. Then it must contain at least one edge.
        \end{quote}
        there is some ambiguity about whether ``it'' refers to $G$ or to the complement of
        $G$. The second sentence is better written as ``Then G must contain at least one
        edge''.

    \item[Use Proper Notation:] There are many notational conventions in mathematics.
        You need to follow the accepted conventions when using notation. For example, A
        summation or integral symbol always needs something to act on. The expressions 
        \[ \sum_{i=1}^n \qquad \qquad \int_a^b \]
        by themselves are meaningless.  The expressions
        \[ \sum_{i=1}^n a_n \qquad \qquad \int_a^b f(x) dx \]
        have well-understood meanings.

        An another example,
        \[ \underbrace{\lim_{h \to 0} = \frac{2x+h}{2}}_{\text{incorrect!}} = \frac{2x}{2} = x \]
        is incorrect.  It should be written
        \[ \lim_{h \to 0} \frac{2x+h}{2} = \frac{2x}{2} = x \]


    \item[Parenthesis are important:] Parenthesis show the grouping of terms, and the
        omission of parenthesis can lead to much unneeded confusion.  For example,
        \[ x^2 + 5 \cdot x-3 \quad \text{ is very different than } \quad \left( x^2 + 5
            \right) \cdot \left( x-3 \right). \]
        This is very important in differentiation and summation notation:
        \[ \frac{d}{dx} \sin(x) + x^2 \quad \text{is not the same as} \quad \frac{d}{dx}
            \left( \sin(x) + x^2 \right) \]
        \[ \sum_{k=1}^n 2k+3 \quad \text{is not the same as} \quad \sum_{k=1}^n \left(
            2k+3 \right) \]

    \item[Label and reference equations:] When you need to refer to an equation later it is
        common practice to label the equation with a number and then to refer to this
        equation by that number.  This avoids ambiguity and gives the reader a better
        chance at understanding what you're writing. Furthermore, avoid using words like
        ``below'' and ``above'' since the reader doesn't really know where to look. One
        implication to this style of referencing is that you should never reference an
        equation before you define it. \\
        {\bf Incorrect:}
        \begin{quote}
            In the equation below we consider the domain $x \in (-1,1)$
            \[ f(x) = \sum_{j=1}^\infty \frac{x^n}{n!}. \]
        \end{quote}
        {\bf Correct:}
        \begin{quote}
            Consider the summation
            \begin{flalign} f(x) = \sum_{j=1}^\infty \frac{x^n}{n!}.\label{eqn:sample_eqn} \end{flalign}
            In \eqref{eqn:sample_eqn} we are assuming the domain $x \in (-1,1)$
        \end{quote}

    \item[``Timesing'':] The act of multiplication should not be called ``timesing'' as in
        ``I can times 3 and 5 to get 15''. The correct version of this sentence is ``I can
        multiply 3 and 5 to get 15''.  The phrase ``3 times 5 is 15'', on the other hand,
        is correct and is likely the root of the confusion.  The mathematical operation
        being performed is not called ``timesing''.  It seems as if this is an
        unfortunate carry over from childhood when a child hears ``3 times 5'', sees ``$3
        \times 5$'', and then incorrectly associates the symbol ``$\times$'' with the word
        multiply in the statement ``I can multiply 3 and 5 to get 15''. 
        
\end{description}


\subsection{Mathematical Vocabulary}
\begin{description}
    \item[Function:] The word function can be used to refer just to the name of a
        function, such as  ``The function $s(t)$ gives the position of the particle as a
        function of time.'' Or function can refer to both the function name and the rule
        that describes the function. For example, we could elaborate and say, ``The
        function $s(t) = t2 – 3t$ gives the position of the particle as a function of
        time.'' Notice that both times the word function is used twice, where the second
        usage is describing the mathematical nature of the relationship between time and
        position. (Remember that if position can be described as a function of time, then
        the position can be uniquely determined from the time.)
    \item[Equation:] To begin with, an equation must have an equal sign ($=$), but just
        having an equal sign isn't enough to deserve the name equation. Generally, an
        equation is something that will be used to solve for a particular variable, and/or
        it expresses a relationship between variables. So you might say, ``We solved the
        equation $x + y = 5$ for $x$ to find that $x = 5 – y$,'' or you might say ``The
        relationship between the variables can be expressed with the following equation:
        $xy = 2z$.''

    \item[Formula:] A formula might in fact be an equation or even a function, but
        generally the word formula is used when you are going to substitute numbers for
        some or all of the variables.  For example, we might say, ``The formula for the
        area of a circle is $A = \pi r^2$.  Since $r = 2$ in this case, we find $A = \pi
        2^2 = 4\pi$.''  The bottom line: If you're going to use algebra to solve for a
        variable, call it an equation. If you’re going to use it exactly as it is and just
        put in numbers for the variables, then call it a formula.

    \item[Definition:] A definition might be any of the above, but it is specifically
        being used to define a new term. For example, the definition of the derivative of
        a function $f$ at a point a is  
        \[ f'(a) = \lim_{h \to 0} \frac{f(a+h)-f(a)}{h}. \]
        Now this does give us a formula to use to compute the derivative, but we prefer to
        call this particular formula a definition to highlight the fact that this is what
        we have chosen the word derivative to mean.

    \item[Expression:] The word expression is used when there isn't an equal sign. You
        probably won't need this word very often, but it is used like this: ``The
        factorization of the expression $x^2 – x – 6$ is $(x – 3)(x + 2)$.''

    \item[Solve/Evaluate:] Equations are solved, whereas functions are evaluated. So you
        would say, ``We solved the equation for $x$,'' but you would say ``We evaluated
        the function at $x = 5$ and found the function value to be $26$.''

    \item[Add Subtract vs Plus Minus:] The word subtract is used when discussing what
        needs to be done: ``Subtract two from five to get three.'' Add is used similarly:
        ``Add two and five to get seven.'' Minus is used when reading a mathematical
        equation or expression. For example, the equation $x – y = 5$ would be read as
        ``$x$ minus $y$ is equal to five''. Plus is used similarly. So the equation $x + y
        = 5$ would be read as ``$x$ plus $y$ is equal to five''. Some things we don't say
        are ``We plus 2 and 5 to get 7'' or ``We minus $x$ from both sides of the
        equation.''

    \item[Number/Amount:] The word number is used when referring to discrete items, such
        as ``there were a large number of cougars'', or ``there are a large number of
        books on my shelf''. The word amount is used when referring to something that
        might come in a pile, such as ``that is a huge amount of sand!'' or, ``I only use
        a small amount of salt when I cook''.

    \item[Many/Much:] These words are used in much the same way as number and amount, with
        many in place of number and much in place of amount.  For example, we might say,
        ``There aren't as many cougars here as before'', or ``I don't use as much salt as
        you do.''

    \item[Fewer/Less:] These are the diminutive analogues of many and much. So, ``There
        are fewer cougars here than before'', or ``You use less salt than I do.''
\end{description}






\section{Sensitivity Analysis}
\noindent (This section is paraphrased partly from Dianne O'Leary's book \emph{Computing
in Science \& Engineering} and partly from Mark Meerschaert's text \emph{Mathematical
Modeling}.) 

In contrast to to classroom exercises, the real world rarely presents us with a problem in
which the data is known with absolute certainty.  Some parameters (such as $\pi$) we can
define with certainty, and others (such as Planck's constant $\hslash$) we know to high
precision, but most data is measured and therefore contains measurement error.  

Thus, what we really solve isn't the problem we want, but some {\it nearby} problem, and in
addition to reporting the computed solution we really need to report a bound on either
\begin{itemize}
    \item the difference between the true solution and the computed solution, or
    \item the difference between the problem we solved and the problem we wanted to solve. 
\end{itemize}
This need occurs throughout computational science.  For example,
\begin{itemize}
    \item If we compute the resonant frequencies of a model of a building, we want to know
        how these frequencies change if the load within the building changes.
    \item If we compute the stresses on a bridge, we want to know how sensitive these
        values are to changes that might occur as the bridge ages.
    \item If we develop a model for our data and fit the parameters using least squares,
        we want to know how much the parameters would change if the data were wiggled
        within the uncertainty limits.
\end{itemize}

One of the best ways to measure the sensitivity of a parameter $k$ on an output $x$ is to
measure the ratio between the relative change in $x$ to the relative change in $k$.  That
is, one measure of sensitivity is
\[ S = \Big| \left( \frac{\Delta x}{x} \right) \Big/ \left( \frac{\Delta k}{k} \right)
    \Big|. \]
Simplifying a bit gives
\[ S = \Big| \left( \frac{\Delta x}{\Delta k} \right) \cdot \left( \frac{k}{x(k)} \right)
    \Big| \]
where we are now explicitly stating that the output $x$ is a function of $k$: $x=x(k)$.  
Taking $\Delta k = \delta$ as well as taking $\Delta x = x(x\pm\delta)-x(k)$ we can rewrite
one more time to give
\[ S = \Big|\left( \frac{x(k\pm\delta)-x(k)}{\delta} \right) \cdot \left( \frac{k}{x(k)}
    \right) \Big|.  \]
Notice that we could take the change of $x$ by increasing $k$ by $\delta$ or
by decreasing $k$ by $\delta$.

The value of $\delta$ is related to known or estimated information about how the parameter
varies.  It is likely that $k$ is a value from some
statistical distribution (like a normal distribution) with an approximately known or
estimated standard deviation. The value of $\delta$ should be related to the standard
deviation and some basic statistics can be used to choose the $\delta$ for your
sensitivity analysis. Recall that if you sample a parameter from the normal distribution then
\begin{itemize}
    \item roughly 68\% of the sampled parameters will be within 1 standard deviation of
        the mean of the normal distribution, and 
    \item roughly 95\% of the sampled parameters will be within $1.96$ standard
        deviations of the mean of the normal distribution.
\end{itemize}
This means that if $k$ comes from a normal distribution then a very typical choice for
$\delta$ is $1.96$ times the value of the estimated standard deviation (up or down from
$k$).  If, on the other hand, the values of the parameter are uniformly distributed then
$\delta$ can be chosen as the maximum estimated deviation from the mean of the
distribution (up or down from $k$).  

Generally:
\begin{itemize}
    \item If the value of $S$ is approximately 1 then the relative changes are approximately the
        same and the output is not very sensitive to changes in the parameter.  
    \item If the value of $S$ is less than 1 then the relative changes in the output are
        less than the changes in the parameter and the output is not sensitive to changes
        in the parameter.  
    \item Finally, if the value of $S$ is larger than 1 then the relative changes in the
        output are greater than the changes in the parameter and the output is considered
        sensitive to changes in the parameter.
\end{itemize}

\section{Example of Sensitivity Analysis:}
Let's do a more specific example.  If we are analyzing the differential equation $P' = kP$
and we estimate that the growth rate is normally distributed with sample mean $k \approx
0.009$ and a standard deviation of $\sigma \approx 0.001$, then we can estimate the
sensitivity of the time needed for the population to double as a function of the growth
rate. In this case, the {\it doubling time} is the output and the {\it growth rate} is the
parameter of interest.

The analytic solution to the differential equation is $P(t) = P_0 e^{kt}$ and the
population doubling can be found by solving $2P_0 = P_0 e^{kT_d}$ where $T_d$ is the time
to double the population.  Using some basic algebra we see that the doubling time as a
function of the growth rate is $T_d(k) = \frac{\ln(2)}{k}$. Therefore, to measure the
sensitivity of the doubling time to changes in the parameter $k$ we can take $\delta
= 1.96 \times 0.001 = 0.00196$.  To measure sensitivity in doubling time to an increase in the
growth rate we see that $S$ is given as follows:
\begin{flalign*}
    S &= \Big| \left( \frac{T_d(k+\delta)-T_d(k)}{\delta} \right) \cdot \left(
    \frac{k}{T_d(k)} \right) \Big| \\
    &= \Big| \left(\frac{\frac{\ln(2)}{k+\delta} -
    \frac{\ln(2)}{k}}{\delta}  \right) \cdot \left( \frac{k}{\frac{\ln(2)}{k}} \right)
    \Big| \\
    & = \Big| \left( \frac{k\ln(2) - (k+\delta)\ln(2) }{\delta k(k+\delta)} \right) \cdot \left(
    \frac{k^2}{\ln(2)} \right) \Big|  \\
    &= \Big| \frac{k}{k+\delta} \Big| \\
    &= \frac{0.009}{0.009 + 1.96 \times 0.001} = \frac{0.009}{0.01096} \approx
    0.8212
\end{flalign*}
To measure sensitivity in doubling time to a decrease in the growth rate we see that $S$
is given as\footnote{I'm showing you the algebra here, but it really isn't necessary to
show this level of routine algebra in your papers. Only show the algebra and other
calculations that are necessary for the reader to understand the work that you're doing.}
\begin{flalign*}
    S &= \Big| \left( \frac{T_d(k-\delta)-T_d(k)}{\delta} \right) \cdot \left(
    \frac{k}{T_d(k)} \right) \Big| \\
    &= \Big| \left(\frac{\frac{\ln(2)}{k-\delta} -
    \frac{\ln(2)}{k}}{\delta}  \right) \cdot \left( \frac{k}{\frac{\ln(2)}{k}} \right)
    \Big| \\
    & = \Big| \left( \frac{k\ln(2) - (k-\delta)\ln(2) }{\delta k(k-\delta)} \right) \cdot \left(
    \frac{k^2}{\ln(2)} \right) \Big|  \\
    &= \Big| \frac{k}{k-\delta} \Big| \\
    &\approx \frac{0.009}{0.009 - 1.96 \times 0.001} = \frac{0.009}{0.00704} \approx
    1.2784.
\end{flalign*}

In the present example, the doubling time is not considered to be very sensitive to
increases in the growth rate but it is sensitive to decreases in the growth rate. Given
that in this simple example we are dealing with an exponential decay function this should
also be intuitively {\it obvious}.








