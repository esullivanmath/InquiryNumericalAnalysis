\chapter{MATLAB Basics}\label{app:MATLAB}
In this appendix we'll go through a few of the basics in MATLAB.  This is by no means
meant to be an all-encompassing resource for MATLAB programming.  A few more thorough
resources for MATLAB are listed here.
\begin{itemize}
    \item \href{https://www.mathworks.com/help/pdf_doc/matlab/matlab_prog.pdf}{https://www.mathworks.com/help/pdf\_doc/matlab/matlab\_prog.pdf}
    \item
        \href{https://www.mathworks.com/products/matlab/examples.html}{https://www.mathworks.com/products/matlab/examples.html}
    \item
        \href{https://en.wikibooks.org/wiki/MATLAB_Programming}{https://en.wikibooks.org/wiki/MATLAB\_Programming}
    \item
        \href{http://gribblelab.org/scicomp/scicomp.pdf}{http://gribblelab.org/scicomp/scicomp.pdf}
        (this is a personal favorite)
\end{itemize}

In this appendix we'll give examples of some of the more common coding practices that the
reader will run into while working through the exercises and problems in these notes.  

\section{Vectors and Matrices}
\begin{example}
    Write the vectors $\bv = \begin{pmatrix} 1 \\ 2 \\ 3 \end{pmatrix} \text{ and } \bw =
        \begin{pmatrix} 4 & 5 & 6 & 7 \end{pmatrix}$ using MATLAB.\\
    {\bf Solution:}
    \begin{lstlisting}
    v = [1 ; 2 ; 3]
    w = [4 , 5 , 6 , 7]
    w = 4:7  % this is shorthand for writing a sequence as a row vector
    \end{lstlisting}
\end{example}

\begin{example}
    Consider the matrices and vectors
    \[ A = \begin{pmatrix} 1 & 2 & 3 \\ 4 & 5 & 6 \\ 7 & 8 & 0 \end{pmatrix} \quad B =
            \begin{pmatrix} 3 & 5 & 7 \\ 9 & 1 & 3 \\ 5 & 7 & 11 \end{pmatrix} \quad
                \text{and} \quad \bb = \begin{pmatrix} 4 & 3 & -1 \end{pmatrix} \]
    \begin{itemize}
        \item Calculate the product $AB$ using regular matrix multiplication
\begin{lstlisting}
A = [1 , 2 , 3; 
    4 , 5 , 6;
    7 , 8 , 0]
B = [3 , 5 , 7;
    9 , 1 , 3;
    5 , 7 , 11]
Product = A*B
\end{lstlisting}
        \item Calculate the element-by-element multiplication of $A$ and $B$
\begin{lstlisting}
ElementWiseProduct = A .* B
\end{lstlisting}
        \item Calculate the inverse of $A$
\begin{lstlisting}            
Ainv = A^(-1)
Ainv = inverse(A) % alternative
\end{lstlisting}            
        \item Calculate the transpose of $B$
\begin{lstlisting}
Atranspose = transpose(A)
% or as an alternative: 
Atranspose = A'  % actually the conjugate transpose but if A is real then ok
\end{lstlisting}
        \item Solve the system of equations $A\bx = \bb$
\begin{lstlisting}
b = [4 ; 3 ; -1]
x = A \ b
\end{lstlisting}
    \end{itemize}
\end{example}

\begin{example}
    Code for a matrix of zeros
\begin{lstlisting}
Z = zeros(5,5) % 5 x 5 matrix of all zeros
\end{lstlisting}
\end{example}

\begin{example}
    Code for an identity matrix
\begin{lstlisting}
Ident = eye(5,5) % 5 x 5 identity matrix
\end{lstlisting}
\end{example}

\begin{example}
    Code for random matrices.
    \begin{itemize}
        \item random matrix from a uniform distribution on $[0,1]$
\begin{lstlisting}
R = rand(5,5) % random 5 x 5 matrix
\end{lstlisting}
        \item random matrix from the standard normal distribution
\begin{lstlisting}
R = randn(5,5) % random 5 x 5 matrix
\end{lstlisting}
    \end{itemize}
\end{example}

\begin{example}
    A linearly spaced sequence
\begin{lstlisting}
List = linspace(0,10,100)
% a list of 100 equally spaced numbers from 0 to 10
\end{lstlisting}
\end{example}

\section{Looping}
A loop is used when a process needs to be repeated several times.  

\subsection{For Loops}
A \texttt{for loop} is code that repeats across a pre-defined sequence.
\begin{example}
    Write a loop that produces the squares of the first 10 integers.
\begin{lstlisting}
for j = 1:10
    j^2
end
\end{lstlisting}
The output of this code will be
\begin{verbatim}
1
4
9
16
25
36
49
64
81
100
\end{verbatim}
\end{example}

\begin{example}
    Plot the functions $f(x) = \sin(kx)$ for $k=1, 1.5, 2, 2.5, \ldots, 5$ on the domain $x \in
    [0,2\pi]$.
\begin{lstlisting}
x = linspace(0,2*pi,1000);
for k = 1:0.5:5
    plot(x , sin(k*x) )
    hold on
end
\end{lstlisting}
\end{example}




\subsection{The While Loop}
A \texttt{while loop} is a process that only repeats while a conditional statement is
true.  Be careful with \texttt{while loops} since it is possible to create a loop that
runs forever.

\begin{example}
    Build the Fibonnaci sequence up until the last term is greater than 1000.
\begin{lstlisting}
F(1) = 1; % first term
F(2) = 1; % second term
n = 3;
while F(end)<1000
    F(n) = F(n-1) + F(n-2);
    n=n+1;
end
\end{lstlisting}
\end{example}

\begin{example}
    An example of a while loop that runs forever.
\begin{lstlisting}
a = 1;
while a>0
    a=a+1;
end
\end{lstlisting}
\end{example}


\begin{example}
    An example of a while loop that runs forever but with a failsafe step that stops the
    loop after 1000 steps.
\begin{lstlisting}
a = 1;
counter=1;
while a>0
    a=a+1;
    if counter >= 1000
        break
    end
    counter = counter+1;
end
\end{lstlisting}
\end{example}

\section{Conditional Statements}
Conditional statements are used to check if something is true or false.  The output of a
conditional statement is a boolean value; true (1) or false (0).  

\subsection{If Statements}
\begin{example}
    Loop over the integers up to 100 and output only the multiples of three.
\begin{lstlisting}
for j = 1:100
    if mod(j,3) == 0
        j
    end
end
\end{lstlisting}
\end{example}

\begin{example}
    Check the signs of two function values and determine if they are opposite.
\begin{lstlisting}
f = @(x) x^3*(x-3);
a = 2;
b = 4;
if f(a)*f(b) < 0
    fprintf('The function values are opposite sign\n')
elseif f(a)*f(b) >0
    fprintf('The function values are the same sign\n')
else
    fprintf('The function values are both zero\n')
end
\end{lstlisting}
\end{example}



\subsection{Case-Switch Statements}
\begin{example}
    Evaluate over several cases.
\begin{lstlisting}
n = 3
switch n
    case 1 % if n == 1
        fprintf('n is 1\n')
    case 2 % if n == 2
        fprintf('n is 2\n')
    case 3 % if n == 3
        fprintf('n is 3\n')
end
\end{lstlisting}
\end{example}

\section{Functions}
A mathematical function has a single output for every input, and in
some sense a computer function is the same: one single executed process for each
collection of inputs.  

\begin{example}
    Define the function $f(x) = \sin(x^2)$ so that it can accept any type of input
    (symbol, number, or list of numbers).
\begin{lstlisting}
f = @(x) sin(x.^2) % defines the function
f(3) % evaluates the function at x=3
x=linspace(0,pi,100);
f(x) % evaluates f at 100 points equally spaced from 0 to pi
\end{lstlisting}
\end{example}

\begin{example}
    Write a computer function that accepts two numbers as inputs and outputs the sum plus
    the product of the two numbers. \\
    First write a file with the following contents.
\begin{lstlisting}
function MyOutput = MyFunctionName(a,b)
    MyOutput = a + b + a*b;
end
\end{lstlisting}
Be sure that the file name is the same as the function name.\\
Then you can call the function by name in a script or another function.
\begin{lstlisting}
SumPlusProduct = MyFunctionName(3,4)
\end{lstlisting}
which will output the number $19$.
\end{example}


\begin{example}
    Write a function with three inputs that outputs the sum of the three.  The third input
    should be optional and the default should be set to 5.
\begin{lstlisting}
function AwesomeOutput = SumOfThree(a,b,c)
    if nargin < 3
        c = 5;
    end
    AwesomeOutput = a+b+c;
end
\end{lstlisting}
You can call this function with
\begin{lstlisting}
SumOfThree(17,23)
\end{lstlisting}
which will output $17+25+5 = 47$.  Notice that the third input was left off and a 5 was
used in its place.
\end{example}


\section{Plotting}
In numerical analysis we are typically plotting numerically computed lists of numbers so
as such we will give a few examples of this type of plotting here.  We will not, however,
give examples of symbolic plotting.

The \mcode{plot} command in MATLAB accepts a list of $x$ values followed by a list of $y$
values then followed by color and symbol options.\\
\mcode{plot(xlist , ylist , color options)}

\begin{example}
    Plot the function $f(x) = \sin(x^2)$ on the interval $[0,2\pi]$ with 1000 equally
    spaced points.  Make the plot color blue.
\begin{lstlisting}
x = linspace(0,2*pi,1000);
f = @(x) sin(x.^2);
plot(x , f(x) , 'b')
\end{lstlisting}
Alternatively
\begin{lstlisting}
x = linspace(0,2*pi,1000);
y = sin(x.^2);
plot(x, y, 'b')
\end{lstlisting}
\end{example}


\begin{example}
    Make a $2\times 2$ array of 4 plots of $f(x) = \sin(k x^2)$ for $k=1, 2, 3, 4$.
\begin{lstlisting}
x = linspace(0,2*pi,1000);
for k=1:4
    subplot(2,2,k)
    plot(x , sin(k*x.^2) , 'b')
end
\end{lstlisting}
\end{example}


\begin{example}
    Plot $f(x) = \sin(kx^2)$ for $k=1, 2, \ldots, 10$ all on the same plot.
\begin{lstlisting}
x = linspace(0,2*pi,1000);
for k=1:10
    plot(x, sin(k*x.^2))
    hold on % this holds the figure window open so you can write on top of it
end
\end{lstlisting}
\end{example}


\begin{example}
    Plot the function $f(x) = e^{-x} \sin(x)$ and put a mark at the local max at $x =
    \pi/4$.
\begin{lstlisting}
x = linspace(0,2*pi,1000); % set up the domain
f = @(x) exp(-x) .* sin(x);
plot(x,f(x),'b',pi/4,f(pi/4),'ro')
\end{lstlisting}
\end{example}

\section{Animations}

\begin{example}
    Plot $f(x) = \sin(kx^2)$ for $k=1$ to $k= 10$ by small increments with a short pause
    in between each step. 
\begin{lstlisting}
x = linspace(0,2*pi,1000);
for k=1:0.01:10 % 1 to 10 by 0.01
    plot(x, sin(k*x.^2))
    hold on % this holds the figure window open so you can write on top of it
    drawnow % draws the plot
    % the last line gives the illusion of animation
end
\end{lstlisting}
\end{example}



